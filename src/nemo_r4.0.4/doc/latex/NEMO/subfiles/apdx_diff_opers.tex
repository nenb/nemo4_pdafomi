\documentclass[../main/NEMO_manual]{subfiles}

\begin{document}

\chapter{Diffusive Operators}
\label{apdx:DIFFOPERS}

\thispagestyle{plain}

\chaptertoc

\paragraph{Changes record} ~\\

{\footnotesize
  \begin{tabularx}{\textwidth}{l||X|X}
    Release & Author(s) & Modifications \\
    \hline
    {\em   4.0} & {\em ...} & {\em ...} \\
    {\em   3.6} & {\em ...} & {\em ...} \\
    {\em   3.4} & {\em ...} & {\em ...} \\
    {\em <=3.4} & {\em ...} & {\em ...}
  \end{tabularx}
}

\clearpage

%% =================================================================================================
\section{Horizontal/Vertical $2^{nd}$ order tracer diffusive operators}
\label{sec:DIFFOPERS_1}

%% =================================================================================================
\subsubsection*{In z-coordinates}

In $z$-coordinates, the horizontal/vertical second order tracer diffusion operator is given by:
\begin{align}
  \label{eq:DIFFOPERS_1}
  &D^T = \frac{1}{e_1 \, e_2}      \left[
    \left. \frac{\partial}{\partial i} \left( 	\frac{e_2}{e_1}A^{lT} \;\left. \frac{\partial T}{\partial i} \right|_z   \right)   \right|_z      \right.
    \left.
    + \left. \frac{\partial}{\partial j} \left(  \frac{e_1}{e_2}A^{lT} \;\left. \frac{\partial T}{\partial j} \right|_z   \right)   \right|_z      \right]
    + \frac{\partial }{\partial z}\left( {A^{vT} \;\frac{\partial T}{\partial z}} \right)
\end{align}

%% =================================================================================================
\subsubsection*{In generalized vertical coordinates}

In $s$-coordinates, we defined the slopes of $s$-surfaces, $\sigma_1$ and $\sigma_2$ by \autoref{eq:SCOORD_s_slope} and
the vertical/horizontal ratio of diffusion coefficient by $\epsilon = A^{vT} / A^{lT}$.
The diffusion operator is given by:

\begin{equation}
  \label{eq:DIFFOPERS_2}
  D^T = \left. \nabla \right|_s \cdot
  \left[ A^{lT} \;\Re \cdot \left. \nabla \right|_s T  \right] \\
  \;\;\text{where} \;\Re =\left( {{
        \begin{array}{*{20}c}
          1 \hfill & 0 \hfill & {-\sigma_1 } \hfill \\
          0 \hfill & 1 \hfill & {-\sigma_2 } \hfill \\
          {-\sigma_1 } \hfill & {-\sigma_2 } \hfill & {\varepsilon +\sigma_1
                                                      ^2+\sigma_2 ^2} \hfill \\
        \end{array}
      }} \right)
\end{equation}
or in expanded form:
\begin{align*}
  {
  \begin{array}{*{20}l}
    D^T= \frac{1}{e_1\,e_2\,e_3 } & \left\{ \quad \quad \frac{\partial }{\partial i}  \left. \left[  e_2\,e_3 \, A^{lT}
                               \left( \  \frac{1}{e_1}\; \left. \frac{\partial T}{\partial i} \right|_s
                                       -\frac{\sigma_1 }{e_3 } \; \frac{\partial T}{\partial s} \right) \right]  \right|_s  \right. \\
        &  \quad \  +   \            \left.   \frac{\partial }{\partial j}  \left. \left[  e_1\,e_3 \, A^{lT}
                               \left( \ \frac{1}{e_2 }\; \left. \frac{\partial T}{\partial j} \right|_s
                                       -\frac{\sigma_2 }{e_3 } \; \frac{\partial T}{\partial s} \right) \right]  \right|_s  \right. \\
        &  \quad \  +   \           \left.  e_1\,e_2\, \frac{\partial }{\partial s}  \left[ A^{lT} \; \left(
	                  -\frac{\sigma_1 }{e_1 } \; \left. \frac{\partial T}{\partial i} \right|_s
	                  -\frac{\sigma_2 }{e_2 } \; \left. \frac{\partial T}{\partial j} \right|_s
                          +\left( \varepsilon +\sigma_1^2+\sigma_2 ^2 \right) \; \frac{1}{e_3 } \; \frac{\partial T}{\partial s} \right) \; \right] \;  \right\} .
  \end{array}
          }
\end{align*}

\autoref{eq:DIFFOPERS_2} is obtained from \autoref{eq:DIFFOPERS_1} without any additional assumption.
Indeed, for the special case $k=z$ and thus $e_3 =1$,
we introduce an arbitrary vertical coordinate $s = s (i,j,z)$ as in \autoref{apdx:SCOORD} and
use \autoref{eq:SCOORD_s_slope} and \autoref{eq:SCOORD_s_chain_rule1}.
Since no cross horizontal derivative $\partial _i \partial _j $ appears in \autoref{eq:DIFFOPERS_1},
the ($i$,$z$) and ($j$,$z$) planes are independent.
The derivation can then be demonstrated for the ($i$,$z$)~$\to$~($j$,$s$) transformation without
any loss of generality:

\begin{align*}
  {
  \begin{array}{*{20}l}
    D^T&=\frac{1}{e_1\,e_2} \left. {\frac{\partial }{\partial i}\left( {\frac{e_2}{e_1}A^{lT}\;\left. {\frac{\partial T}{\partial i}} \right|_z } \right)} \right|_z
         +\frac{\partial }{\partial z}\left( {A^{vT}\;\frac{\partial T}{\partial z}} \right) \\ \\
         %
       &=\frac{1}{e_1\,e_2 }\left[ {\left. {\;\frac{\partial }{\partial i}\left( {\frac{e_2}{e_1}A^{lT}\;\left( {\left. {\frac{\partial T}{\partial i}} \right|_s
         -\frac{e_1\,\sigma_1 }{e_3 }\frac{\partial T}{\partial s}} \right)} \right)} \right|_s } \right. \\
       & \qquad \qquad \left. { -\frac{e_1\,\sigma_1 }{e_3 }\frac{\partial }{\partial s}\left( {\frac{e_2 }{e_1 }A^{lT}\;\left. {\left( {\left. {\frac{\partial T}{\partial i}} \right|_s -\frac{e_1 \,\sigma_1 }{e_3 }\frac{\partial T}{\partial s}} \right)} \right|_s } \right)\;} \right]
         \shoveright{ +\frac{1}{e_3 }\frac{\partial }{\partial s}\left[ {\frac{A^{vT}}{e_3 }\;\frac{\partial T}{\partial s}} \right]}  \qquad \qquad \qquad \\ \\
         %
       &=\frac{1}{e_1 \,e_2 \,e_3 }\left[ {\left. {\;\;\frac{\partial }{\partial i}\left( {\frac{e_2 \,e_3 }{e_1 }A^{lT}\;\left. {\frac{\partial T}{\partial i}} \right|_s } \right)} \right|_s -\left. {\frac{e_2 }{e_1}A^{lT}\;\frac{\partial e_3 }{\partial i}} \right|_s \left. {\frac{\partial T}{\partial i}} \right|_s } \right. \\
       &  \qquad \qquad \quad \left. {-e_3 \frac{\partial }{\partial i}\left( {\frac{e_2 \,\sigma_1 }{e_3 }A^{lT}\;\frac{\partial T}{\partial s}} \right)} \right|_s -e_1 \,\sigma_1 \frac{\partial }{\partial s}\left( {\frac{e_2 }{e_1 }A^{lT}\;\left. {\frac{\partial T}{\partial i}} \right|_s } \right) \\
       &  \qquad \qquad \quad \shoveright{ -e_1 \,\sigma_1 \frac{\partial }{\partial s}\left( {-\frac{e_2 \,\sigma_1 }{e_3 }A^{lT}\;\frac{\partial T}{\partial s}} \right)\;\,\left. {+\frac{\partial }{\partial s}\left( {\frac{e_1 \,e_2 }{e_3 }A^{vT}\;\frac{\partial T}{\partial s}} \right)\quad} \right] }\\
  \end{array}
  } 		\\
  %
  {
  \begin{array}{*{20}l}
    \intertext{Noting that $\frac{1}{e_1} \left. \frac{\partial e_3 }{\partial i} \right|_s = \frac{\partial \sigma_1 }{\partial s}$, this becomes:}
    %
    D^T & =\frac{1}{e_1\,e_2\,e_3 }\left[ {\left. {\;\;\;\frac{\partial }{\partial i}\left( {\frac{e_2\,e_3 }{e_1}\,A^{lT}\;\left. {\frac{\partial T}{\partial i}} \right|_s } \right)} \right|_s \left. -\, {e_3 \frac{\partial }{\partial i}\left( {\frac{e_2 \,\sigma_1 }{e_3 }A^{lT}\;\frac{\partial T}{\partial s}} \right)} \right|_s } \right. \\
    & \qquad \qquad \quad -e_2 A^{lT}\;\frac{\partial \sigma_1 }{\partial s}\left. {\frac{\partial T}{\partial i}} \right|_s -e_1 \,\sigma_1 \frac{\partial }{\partial s}\left( {\frac{e_2 }{e_1 }A^{lT}\;\left. {\frac{\partial T}{\partial i}} \right|_s } \right) \\
    & \qquad \qquad \quad\shoveright{ \left. { +e_1 \,\sigma_1 \frac{\partial }{\partial s}\left( {\frac{e_2 \,\sigma_1 }{e_3 }A^{lT}\;\frac{\partial T}{\partial s}} \right)+\frac{\partial }{\partial s}\left( {\frac{e_1 \,e_2 }{e_3 }A^{vT}\;\frac{\partial T}{\partial s}} \right)\;\;\;} \right] }\\
    \\
    &=\frac{1}{e_1 \,e_2 \,e_3 } \left[ {\left. {\;\;\;\frac{\partial }{\partial i} \left( {\frac{e_2 \,e_3 }{e_1 }A^{lT}\;\left. {\frac{\partial T}{\partial i}} \right|_s } \right)} \right|_s \left. {-\frac{\partial }{\partial i}\left( {e_2 \,\sigma_1 A^{lT}\;\frac{\partial T}{\partial s}} \right)} \right|_s } \right. \\
    & \qquad \qquad \quad \left. {+\frac{e_2 \,\sigma_1 }{e_3}A^{lT}\;\frac{\partial T}{\partial s} \;\frac{\partial e_3 }{\partial i}}  \right|_s -e_2 A^{lT}\;\frac{\partial \sigma_1 }{\partial s}\left. {\frac{\partial T}{\partial i}} \right|_s \\
    & \qquad \qquad \quad-e_2 \,\sigma_1 \frac{\partial}{\partial s}\left( {A^{lT}\;\left. {\frac{\partial T}{\partial i}} \right|_s } \right)+\frac{\partial }{\partial s}\left( {\frac{e_1 \,e_2 \,\sigma_1 ^2}{e_3 }A^{lT}\;\frac{\partial T}{\partial s}} \right) \\
    & \qquad \qquad \quad\shoveright{ \left. {-\frac{\partial \left( {e_1 \,e_2 \,\sigma_1 } \right)}{\partial s} \left( {\frac{\sigma_1 }{e_3}A^{lT}\;\frac{\partial T}{\partial s}} \right) + \frac{\partial }{\partial s}\left( {\frac{e_1 \,e_2 }{e_3 }A^{vT}\;\frac{\partial T}{\partial s}} \right)\;\;\;} \right]} .
  \end{array}
      } \\
  {
  \begin{array}{*{20}l}
    %
    \intertext{Using the same remark as just above, $D^T$ becomes:}
    %
   D^T &= \frac{1}{e_1 \,e_2 \,e_3 } \left[ {\left. {\;\;\;\frac{\partial }{\partial i} \left( {\frac{e_2 \,e_3 }{e_1 }A^{lT}\;\left. {\frac{\partial T}{\partial i}} \right|_s -e_2 \,\sigma_1 A^{lT}\;\frac{\partial T}{\partial s}} \right)} \right|_s } \right.\;\;\; \\
    & \qquad \qquad \quad+\frac{e_1 \,e_2 \,\sigma_1 }{e_3 }A^{lT}\;\frac{\partial T}{\partial s}\;\frac{\partial \sigma_1 }{\partial s} - \frac {\sigma_1 }{e_3} A^{lT} \;\frac{\partial \left( {e_1 \,e_2 \,\sigma_1 } \right)}{\partial s}\;\frac{\partial T}{\partial s} \\
    & \qquad \qquad \quad-e_2 \left( {A^{lT}\;\frac{\partial \sigma_1 }{\partial s}\left. {\frac{\partial T}{\partial i}} \right|_s +\frac{\partial }{\partial s}\left( {\sigma_1 A^{lT}\;\left. {\frac{\partial T}{\partial i}} \right|_s } \right)-\frac{\partial \sigma_1 }{\partial s}\;A^{lT}\;\left. {\frac{\partial T}{\partial i}} \right|_s } \right) \\
    & \qquad \qquad \quad\shoveright{\left. {+\frac{\partial }{\partial s}\left( {\frac{e_1 \,e_2 \,\sigma_1 ^2}{e_3 }A^{lT}\;\frac{\partial T}{\partial s}+\frac{e_1 \,e_2}{e_3 }A^{vT}\;\frac{\partial T}{\partial s}} \right)\;\;\;} \right] . }
  \end{array}
      } \\
  {
  \begin{array}{*{20}l}
    %
    \intertext{Since the horizontal scale factors do not depend on the vertical coordinate,
    the two terms on the second line cancel, while
    the third line reduces to a single vertical derivative, so it becomes:}
  %
    D^T & =\frac{1}{e_1 \,e_2 \,e_3 }\left[ {\left. {\;\;\;\frac{\partial }{\partial i}\left( {\frac{e_2 \,e_3 }{e_1 }A^{lT}\;\left. {\frac{\partial T}{\partial i}} \right|_s -e_2 \,\sigma_1 \,A^{lT}\;\frac{\partial T}{\partial s}} \right)} \right|_s } \right. \\
    & \qquad \qquad \quad \shoveright{ \left. {+\frac{\partial }{\partial s}\left( {-e_2 \,\sigma_1 \,A^{lT}\;\left. {\frac{\partial T}{\partial i}} \right|_s +A^{lT}\frac{e_1 \,e_2 }{e_3 }\;\left( {\varepsilon +\sigma_1 ^2} \right)\frac{\partial T}{\partial s}} \right)\;\;\;} \right]} \\
    %
    \intertext{In other words, the horizontal/vertical Laplacian operator in the ($i$,$s$) plane takes the following form:}
  \end{array}
  } \\
  %
  {\frac{1}{e_1\,e_2\,e_3}}
  \left( {{
  \begin{array}{*{30}c}
    {\left. {\frac{\partial \left( {e_2 e_3 \bullet } \right)}{\partial i}} \right|_s } \hfill \\
    {\frac{\partial \left( {e_1 e_2 \bullet } \right)}{\partial s}} \hfill \\
  \end{array}}}
  \right)
  \cdot \left[ {A^{lT}
  \left( {{
  \begin{array}{*{30}c}
    {1} \hfill & {-\sigma_1 } \hfill \\
    {-\sigma_1} \hfill & {\varepsilon + \sigma_1^2} \hfill \\
  \end{array}
  }} \right)
  \cdot
  \left( {{
  \begin{array}{*{30}c}
    {\frac{1}{e_1 }\;\left. {\frac{\partial \bullet }{\partial i}} \right|_s } \hfill \\
    {\frac{1}{e_3 }\;\frac{\partial \bullet }{\partial s}} \hfill \\
  \end{array}
  }}       \right) \left( T \right)} \right]
\end{align*}
%\addtocounter{equation}{-2}

%% =================================================================================================
\section{Iso/Diapycnal $2^{nd}$ order tracer diffusive operators}
\label{sec:DIFFOPERS_2}

%% =================================================================================================
\subsubsection*{In z-coordinates}

The iso/diapycnal diffusive tensor $\textbf {A}_{\textbf I}$ expressed in
the ($i$,$j$,$k$) curvilinear coordinate system in which
the equations of the ocean circulation model are formulated,
takes the following form \citep{redi_JPO82}:

\begin{equation}
  \label{eq:DIFFOPERS_3}
  \textbf {A}_{\textbf I} = \frac{A^{lT}}{\left( {1+a_1 ^2+a_2 ^2} \right)}
  \left[ {{
        \begin{array}{*{20}c}
          {1+a_2 ^2 +\varepsilon a_1 ^2} \hfill & {-a_1 a_2 (1-\varepsilon)} \hfill & {-a_1 (1-\varepsilon) } \hfill \\
          {-a_1 a_2 (1-\varepsilon) } \hfill & {1+a_1 ^2 +\varepsilon a_2 ^2} \hfill & {-a_2 (1-\varepsilon)} \hfill \\
          {-a_1 (1-\varepsilon)} \hfill & {-a_2 (1-\varepsilon)} \hfill & {\varepsilon +a_1 ^2+a_2 ^2} \hfill \\
        \end{array}
      }} \right]
\end{equation}
where ($a_1$, $a_2$) are $(-1) \times$ the isopycnal slopes in
($\textbf{i}$, $\textbf{j}$) directions, relative to geopotentials (or
equivalently the slopes of the geopotential surfaces in the isopycnal
coordinate framework):
\[
  a_1 =\frac{e_3 }{e_1 }\left( {\frac{\partial \rho }{\partial i}} \right)\left( {\frac{\partial \rho }{\partial k}} \right)^{-1}
  \qquad , \qquad
  a_2 =\frac{e_3 }{e_2 }\left( {\frac{\partial \rho }{\partial j}}
  \right)\left( {\frac{\partial \rho }{\partial k}} \right)^{-1}
\]
and, as before, $\epsilon = A^{vT} / A^{lT}$.

In practice, $\epsilon$ is small and isopycnal slopes are generally less than $10^{-2}$ in the ocean,
so $\textbf {A}_{\textbf I}$ can be simplified appreciably \citep{cox_OM87}. Keeping leading order terms\footnote{Apart from the (1,0)
and (0,1) elements which are set to zero. See \citet{griffies_bk04}, section 14.1.4.1 for a discussion of this point.}:
\begin{subequations}
  \label{eq:DIFFOPERS_4}
  \begin{equation}
    \label{eq:DIFFOPERS_4a}
    {\textbf{A}_{\textbf{I}}} \approx A^{lT}\;\Re\;\text{where} \;\Re =
    \left[ {{
          \begin{array}{*{20}c}
            1 \hfill & 0 \hfill & {-a_1 } \hfill \\
            0 \hfill & 1 \hfill & {-a_2 } \hfill \\
            {-a_1 } \hfill & {-a_2 } \hfill & {\varepsilon +a_1 ^2+a_2 ^2} \hfill \\
          \end{array}
        }} \right],
  \end{equation}
  and the iso/dianeutral diffusive operator in $z$-coordinates is then
  \begin{equation}
    \label{eq:DIFFOPERS_4b}
    D^T = \left. \nabla \right|_z \cdot
    \left[ A^{lT} \;\Re \cdot \left. \nabla \right|_z T  \right]. \\
  \end{equation}
\end{subequations}

Physically, the full tensor \autoref{eq:DIFFOPERS_3} represents strong isoneutral diffusion on a plane parallel to
the isoneutral surface and weak dianeutral diffusion perpendicular to this plane.
However,
the approximate `weak-slope' tensor \autoref{eq:DIFFOPERS_4a} represents strong diffusion along the isoneutral surface,
with weak \emph{vertical} diffusion -- the principal axes of the tensor are no longer orthogonal.
This simplification also decouples the ($i$,$z$) and ($j$,$z$) planes of the tensor.
The weak-slope operator therefore takes the same form, \autoref{eq:DIFFOPERS_4}, as \autoref{eq:DIFFOPERS_2},
the diffusion operator for geopotential diffusion written in non-orthogonal $i,j,s$-coordinates.
Written out explicitly,

\begin{multline}
  \label{eq:DIFFOPERS_ldfiso}
  D^T=\frac{1}{e_1 e_2 }\left\{
    {\;\frac{\partial }{\partial i}\left[ {A_h \left( {\frac{e_2}{e_1}\frac{\partial T}{\partial i}-a_1 \frac{e_2}{e_3}\frac{\partial T}{\partial k}} \right)} \right]}
    {+\frac{\partial}{\partial j}\left[ {A_h \left( {\frac{e_1}{e_2}\frac{\partial T}{\partial j}-a_2 \frac{e_1}{e_3}\frac{\partial T}{\partial k}} \right)} \right]\;} \right\} \\
  \shoveright{+\frac{1}{e_3 }\frac{\partial }{\partial k}\left[ {A_h \left( {-\frac{a_1 }{e_1 }\frac{\partial T}{\partial i}-\frac{a_2 }{e_2 }\frac{\partial T}{\partial j}+\frac{\left( {a_1 ^2+a_2 ^2+\varepsilon} \right)}{e_3 }\frac{\partial T}{\partial k}} \right)} \right]}. \\
\end{multline}

The isopycnal diffusion operator \autoref{eq:DIFFOPERS_4},
\autoref{eq:DIFFOPERS_ldfiso} conserves tracer quantity and dissipates its square.
As \autoref{eq:DIFFOPERS_4} is the divergence of a flux, the demonstration of the first property is trivial, providing that the flux normal to the boundary is zero
(as it is when $A_h$ is zero at the boundary). Let us demonstrate the second one:
\[
  \iiint\limits_D T\;\nabla .\left( {\textbf{A}}_{\textbf{I}} \nabla T \right)\,dv
  = -\iiint\limits_D \nabla T\;.\left( {\textbf{A}}_{\textbf{I}} \nabla T \right)\,dv,
\]
and since
\begin{align*}
  {
  \begin{array}{*{20}l}
    \nabla T\;.\left( {{\mathrm {\mathbf A}}_{\mathrm {\mathbf I}} \nabla T}
    \right)&=A^{lT}\left[ {\left( {\frac{\partial T}{\partial i}} \right)^2-2a_1
             \frac{\partial T}{\partial i}\frac{\partial T}{\partial k}+\left(
             {\frac{\partial T}{\partial j}} \right)^2} \right. \\
           &\qquad \qquad \qquad
             { \left. -\,{2a_2 \frac{\partial T}{\partial j}\frac{\partial T}{\partial k}+\left( {a_1 ^2+a_2 ^2+\varepsilon} \right)\left( {\frac{\partial T}{\partial k}} \right)^2} \right]} \\
           &=A_h \left[ {\left( {\frac{\partial T}{\partial i}-a_1 \frac{\partial
             T}{\partial k}} \right)^2+\left( {\frac{\partial T}{\partial
             j}-a_2 \frac{\partial T}{\partial k}} \right)^2}
             +\varepsilon \left(\frac{\partial T}{\partial k}\right) ^2\right]      \\
           & \geq 0 .
  \end{array}
             }
\end{align*}
%\addtocounter{equation}{-1}
the property becomes obvious.

%% =================================================================================================
\subsubsection*{In generalized vertical coordinates}

Because the weak-slope operator \autoref{eq:DIFFOPERS_4},
\autoref{eq:DIFFOPERS_ldfiso} is decoupled in the ($i$,$z$) and ($j$,$z$) planes,
it may be transformed into generalized $s$-coordinates in the same way as
\autoref{sec:DIFFOPERS_1} was transformed into \autoref{sec:DIFFOPERS_2}.
The resulting operator then takes the simple form

\begin{equation}
  \label{eq:DIFFOPERS_ldfiso_s}
  D^T = \left. \nabla \right|_s \cdot
  \left[ A^{lT} \;\Re \cdot \left. \nabla \right|_s T  \right] \\
  \;\;\text{where} \;\Re =\left( {{
        \begin{array}{*{20}c}
          1 \hfill & 0 \hfill & {-r _1 } \hfill \\
          0 \hfill & 1 \hfill & {-r _2 } \hfill \\
          {-r _1 } \hfill & {-r _2 } \hfill & {\varepsilon +r _1
                                              ^2+r _2 ^2} \hfill \\
        \end{array}
      }} \right),
\end{equation}

where ($r_1$, $r_2$) are $(-1)\times$ the isopycnal slopes in ($\textbf{i}$, $\textbf{j}$) directions,
relative to $s$-coordinate surfaces (or equivalently the slopes of the
$s$-coordinate surfaces in the isopycnal coordinate framework):
\[
  r_1 =\frac{e_3 }{e_1 }\left( {\frac{\partial \rho }{\partial i}} \right)\left( {\frac{\partial \rho }{\partial s}} \right)^{-1}
  \qquad , \qquad
  r_2 =\frac{e_3 }{e_2 }\left( {\frac{\partial \rho }{\partial j}}
  \right)\left( {\frac{\partial \rho }{\partial s}} \right)^{-1}.
\]

To prove \autoref{eq:DIFFOPERS_ldfiso_s} by direct re-expression of \autoref{eq:DIFFOPERS_ldfiso} is straightforward, but laborious.
An easier way is first to note (by reversing the derivation of \autoref{sec:DIFFOPERS_2} from \autoref{sec:DIFFOPERS_1} ) that
the weak-slope operator may be \emph{exactly} reexpressed in non-orthogonal $i,j,\rho$-coordinates as

\begin{equation}
  \label{eq:DIFFOPERS_5}
  D^T = \left. \nabla \right|_\rho \cdot
  \left[ A^{lT} \;\Re \cdot \left. \nabla \right|_\rho T  \right] \\
  \;\;\text{where} \;\Re =\left( {{
        \begin{array}{*{20}c}
          1 \hfill & 0 \hfill &0 \hfill \\
          0 \hfill & 1 \hfill & 0 \hfill \\
          0 \hfill & 0 \hfill & \varepsilon \hfill \\
        \end{array}
      }} \right).
\end{equation}
Then direct transformation from $i,j,\rho$-coordinates to $i,j,s$-coordinates gives
\autoref{eq:DIFFOPERS_ldfiso_s} immediately.

Note that the weak-slope approximation is only made in transforming from
the (rotated,orthogonal) isoneutral axes to the non-orthogonal $i,j,\rho$-coordinates.
The further transformation into $i,j,s$-coordinates is exact, whatever the steepness of the $s$-surfaces,
in the same way as the transformation of horizontal/vertical Laplacian diffusion in $z$-coordinates in
\autoref{sec:DIFFOPERS_1} onto $s$-coordinates is exact, however steep the $s$-surfaces.

%% =================================================================================================
\section{Lateral/Vertical momentum diffusive operators}
\label{sec:DIFFOPERS_3}

The second order momentum diffusion operator (Laplacian) in $z$-coordinates is found by
applying \autoref{eq:MB_lap_vector}, the expression for the Laplacian of a vector,
to the horizontal velocity vector:
\begin{align*}
  \Delta {\textbf{U}}_h
  &=\nabla \left( {\nabla \cdot {\textbf{U}}_h } \right)-
    \nabla \times \left( {\nabla \times {\textbf{U}}_h } \right) \\ \\
  &=\left( {{
    \begin{array}{*{20}c}
      {\frac{1}{e_1 }\frac{\partial \chi }{\partial i}} \hfill \\
      {\frac{1}{e_2 }\frac{\partial \chi }{\partial j}} \hfill \\
      {\frac{1}{e_3 }\frac{\partial \chi }{\partial k}} \hfill \\
    \end{array}
  }} \right)
  -\left( {{
  \begin{array}{*{20}c}
    {\frac{1}{e_2 }\frac{\partial \zeta }{\partial j}-\frac{1}{e_3
    }\frac{\partial }{\partial k}\left( {\frac{1}{e_3 }\frac{\partial
    u}{\partial k}} \right)} \hfill \\
    {\frac{1}{e_3 }\frac{\partial }{\partial k}\left( {-\frac{1}{e_3
    }\frac{\partial v}{\partial k}} \right)-\frac{1}{e_1 }\frac{\partial \zeta
    }{\partial i}} \hfill \\
    {\frac{1}{e_1 e_2 }\left[ {\frac{\partial }{\partial i}\left( {\frac{e_2
    }{e_3 }\frac{\partial u}{\partial k}} \right)-\frac{\partial }{\partial
    j}\left( {-\frac{e_1 }{e_3 }\frac{\partial v}{\partial k}} \right)} \right]}
    \hfill \\
  \end{array}
  }} \right) \\ \\
  &=\left( {{
    \begin{array}{*{20}c}
      {\frac{1}{e_1 }\frac{\partial \chi }{\partial i}-\frac{1}{e_2 }\frac{\partial \zeta }{\partial j}} \\
      {\frac{1}{e_2 }\frac{\partial \chi }{\partial j}+\frac{1}{e_1 }\frac{\partial \zeta }{\partial i}} \\
      0 \\
    \end{array}
  }} \right)
  +\frac{1}{e_3 }
  \left( {{
  \begin{array}{*{20}c}
    {\frac{\partial }{\partial k}\left( {\frac{1}{e_3 }\frac{\partial u}{\partial k}} \right)} \\
    {\frac{\partial }{\partial k}\left( {\frac{1}{e_3 }\frac{\partial v}{\partial k}} \right)} \\
    {\frac{\partial \chi }{\partial k}-\frac{1}{e_1 e_2 }\left( {\frac{\partial ^2\left( {e_2 \,u} \right)}{\partial i\partial k}+\frac{\partial ^2\left( {e_1 \,v} \right)}{\partial j\partial k}} \right)} \\
  \end{array}
  }} \right)
\end{align*}
Using \autoref{eq:MB_div}, the definition of the horizontal divergence,
the third component of the second vector is obviously zero and thus :
\[
  \Delta {\textbf{U}}_h = \nabla _h \left( \chi \right) - \nabla _h \times \left( \zeta \textbf{k} \right) + \frac {1}{e_3 } \frac {\partial }{\partial k} \left( {\frac {1}{e_3 } \frac{\partial {\textbf{ U}}_h }{\partial k}} \right) .
\]

Note that this operator ensures a full separation between
the vorticity and horizontal divergence fields (see \autoref{apdx:INVARIANTS}).
It is only equal to a Laplacian applied to each component in Cartesian coordinates, not on the sphere.

The horizontal/vertical second order (Laplacian type) operator used to diffuse horizontal momentum in
the $z$-coordinate therefore takes the following form:
\begin{equation}
  \label{eq:DIFFOPERS_Lap_U}
  {
    \textbf{D}}^{\textbf{U}} =
  \nabla _h \left( {A^{lm}\;\chi } \right)
  - \nabla _h \times \left( {A^{lm}\;\zeta \;{\textbf{k}}} \right)
  + \frac{1}{e_3 }\frac{\partial }{\partial k}\left( {\frac{A^{vm}\;}{e_3 }
      \frac{\partial {\mathrm {\mathbf U}}_h }{\partial k}} \right) , \\
\end{equation}
that is, in expanded form:
\begin{align*}
  D^{\textbf{U}}_u
  & = \frac{1}{e_1} \frac{\partial \left( {A^{lm}\chi   } \right)}{\partial i}
    -\frac{1}{e_2} \frac{\partial \left( {A^{lm}\zeta } \right)}{\partial j}
    +\frac{1}{e_3} \frac{\partial }{\partial k} \left( \frac{A^{vm}}{e_3} \frac{\partial u}{\partial k} \right)   ,   \\
  D^{\textbf{U}}_v
  & = \frac{1}{e_2 }\frac{\partial \left( {A^{lm}\chi   } \right)}{\partial j}
    +\frac{1}{e_1 }\frac{\partial \left( {A^{lm}\zeta } \right)}{\partial i}
    +\frac{1}{e_3} \frac{\partial }{\partial k} \left( \frac{A^{vm}}{e_3} \frac{\partial v}{\partial k} \right) .
\end{align*}

Note Bene: introducing a rotation in \autoref{eq:DIFFOPERS_Lap_U} does not lead to
a useful expression for the iso/diapycnal Laplacian operator in the $z$-coordinate.
Similarly, we did not found an expression of practical use for
the geopotential horizontal/vertical Laplacian operator in the $s$-coordinate.
Generally, \autoref{eq:DIFFOPERS_Lap_U} is used in both $z$- and $s$-coordinate systems,
that is a Laplacian diffusion is applied on momentum along the coordinate directions.

\subinc{
\clearpage

%% Bibliography
\phantomsection
\addcontentsline{toc}{chapter}{Bibliography}
\lohead{Bibliography} \rehead{Bibliography}
\bibliography{../main/bibliography}

\clearpage

%% Indexes
\phantomsection
\addcontentsline{toc}{chapter}{Indexes}
\lohead{Indexes} \rehead{Indexes}
\printindex[blocks]
\printindex[keys]
\printindex[modules]
\printindex[parameters]
\printindex[subroutines]
}

\end{document}
